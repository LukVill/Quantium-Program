% Options for packages loaded elsewhere
\PassOptionsToPackage{unicode}{hyperref}
\PassOptionsToPackage{hyphens}{url}
%
\documentclass[
]{article}
\usepackage{amsmath,amssymb}
\usepackage{lmodern}
\usepackage{iftex}
\ifPDFTeX
  \usepackage[T1]{fontenc}
  \usepackage[utf8]{inputenc}
  \usepackage{textcomp} % provide euro and other symbols
\else % if luatex or xetex
  \usepackage{unicode-math}
  \defaultfontfeatures{Scale=MatchLowercase}
  \defaultfontfeatures[\rmfamily]{Ligatures=TeX,Scale=1}
  \setmainfont[]{Roboto}
  \setmonofont[]{Consolas}
\fi
% Use upquote if available, for straight quotes in verbatim environments
\IfFileExists{upquote.sty}{\usepackage{upquote}}{}
\IfFileExists{microtype.sty}{% use microtype if available
  \usepackage[]{microtype}
  \UseMicrotypeSet[protrusion]{basicmath} % disable protrusion for tt fonts
}{}
\makeatletter
\@ifundefined{KOMAClassName}{% if non-KOMA class
  \IfFileExists{parskip.sty}{%
    \usepackage{parskip}
  }{% else
    \setlength{\parindent}{0pt}
    \setlength{\parskip}{6pt plus 2pt minus 1pt}}
}{% if KOMA class
  \KOMAoptions{parskip=half}}
\makeatother
\usepackage{xcolor}
\usepackage[margin=1in]{geometry}
\usepackage{color}
\usepackage{fancyvrb}
\newcommand{\VerbBar}{|}
\newcommand{\VERB}{\Verb[commandchars=\\\{\}]}
\DefineVerbatimEnvironment{Highlighting}{Verbatim}{commandchars=\\\{\}}
% Add ',fontsize=\small' for more characters per line
\usepackage{framed}
\definecolor{shadecolor}{RGB}{248,248,248}
\newenvironment{Shaded}{\begin{snugshade}}{\end{snugshade}}
\newcommand{\AlertTok}[1]{\textcolor[rgb]{0.94,0.16,0.16}{#1}}
\newcommand{\AnnotationTok}[1]{\textcolor[rgb]{0.56,0.35,0.01}{\textbf{\textit{#1}}}}
\newcommand{\AttributeTok}[1]{\textcolor[rgb]{0.77,0.63,0.00}{#1}}
\newcommand{\BaseNTok}[1]{\textcolor[rgb]{0.00,0.00,0.81}{#1}}
\newcommand{\BuiltInTok}[1]{#1}
\newcommand{\CharTok}[1]{\textcolor[rgb]{0.31,0.60,0.02}{#1}}
\newcommand{\CommentTok}[1]{\textcolor[rgb]{0.56,0.35,0.01}{\textit{#1}}}
\newcommand{\CommentVarTok}[1]{\textcolor[rgb]{0.56,0.35,0.01}{\textbf{\textit{#1}}}}
\newcommand{\ConstantTok}[1]{\textcolor[rgb]{0.00,0.00,0.00}{#1}}
\newcommand{\ControlFlowTok}[1]{\textcolor[rgb]{0.13,0.29,0.53}{\textbf{#1}}}
\newcommand{\DataTypeTok}[1]{\textcolor[rgb]{0.13,0.29,0.53}{#1}}
\newcommand{\DecValTok}[1]{\textcolor[rgb]{0.00,0.00,0.81}{#1}}
\newcommand{\DocumentationTok}[1]{\textcolor[rgb]{0.56,0.35,0.01}{\textbf{\textit{#1}}}}
\newcommand{\ErrorTok}[1]{\textcolor[rgb]{0.64,0.00,0.00}{\textbf{#1}}}
\newcommand{\ExtensionTok}[1]{#1}
\newcommand{\FloatTok}[1]{\textcolor[rgb]{0.00,0.00,0.81}{#1}}
\newcommand{\FunctionTok}[1]{\textcolor[rgb]{0.00,0.00,0.00}{#1}}
\newcommand{\ImportTok}[1]{#1}
\newcommand{\InformationTok}[1]{\textcolor[rgb]{0.56,0.35,0.01}{\textbf{\textit{#1}}}}
\newcommand{\KeywordTok}[1]{\textcolor[rgb]{0.13,0.29,0.53}{\textbf{#1}}}
\newcommand{\NormalTok}[1]{#1}
\newcommand{\OperatorTok}[1]{\textcolor[rgb]{0.81,0.36,0.00}{\textbf{#1}}}
\newcommand{\OtherTok}[1]{\textcolor[rgb]{0.56,0.35,0.01}{#1}}
\newcommand{\PreprocessorTok}[1]{\textcolor[rgb]{0.56,0.35,0.01}{\textit{#1}}}
\newcommand{\RegionMarkerTok}[1]{#1}
\newcommand{\SpecialCharTok}[1]{\textcolor[rgb]{0.00,0.00,0.00}{#1}}
\newcommand{\SpecialStringTok}[1]{\textcolor[rgb]{0.31,0.60,0.02}{#1}}
\newcommand{\StringTok}[1]{\textcolor[rgb]{0.31,0.60,0.02}{#1}}
\newcommand{\VariableTok}[1]{\textcolor[rgb]{0.00,0.00,0.00}{#1}}
\newcommand{\VerbatimStringTok}[1]{\textcolor[rgb]{0.31,0.60,0.02}{#1}}
\newcommand{\WarningTok}[1]{\textcolor[rgb]{0.56,0.35,0.01}{\textbf{\textit{#1}}}}
\usepackage{graphicx}
\makeatletter
\def\maxwidth{\ifdim\Gin@nat@width>\linewidth\linewidth\else\Gin@nat@width\fi}
\def\maxheight{\ifdim\Gin@nat@height>\textheight\textheight\else\Gin@nat@height\fi}
\makeatother
% Scale images if necessary, so that they will not overflow the page
% margins by default, and it is still possible to overwrite the defaults
% using explicit options in \includegraphics[width, height, ...]{}
\setkeys{Gin}{width=\maxwidth,height=\maxheight,keepaspectratio}
% Set default figure placement to htbp
\makeatletter
\def\fps@figure{htbp}
\makeatother
\setlength{\emergencystretch}{3em} % prevent overfull lines
\providecommand{\tightlist}{%
  \setlength{\itemsep}{0pt}\setlength{\parskip}{0pt}}
\setcounter{secnumdepth}{-\maxdimen} % remove section numbering
\usepackage{fvextra} \DefineVerbatimEnvironment{Highlighting}{Verbatim}{breaklines,commandchars=\\\{\}}
\ifLuaTeX
  \usepackage{selnolig}  % disable illegal ligatures
\fi
\IfFileExists{bookmark.sty}{\usepackage{bookmark}}{\usepackage{hyperref}}
\IfFileExists{xurl.sty}{\usepackage{xurl}}{} % add URL line breaks if available
\urlstyle{same} % disable monospaced font for URLs
\hypersetup{
  pdftitle={Quantium Virtual Internship - Retail Strategy and Analytics - Task 1},
  hidelinks,
  pdfcreator={LaTeX via pandoc}}

\title{Quantium Virtual Internship - Retail Strategy and Analytics -
Task 1}
\author{}
\date{\vspace{-2.5em}}

\begin{document}
\maketitle

\hypertarget{solution-template-for-task-1}{%
\section{Solution template for Task
1}\label{solution-template-for-task-1}}

This file is a solution template for the Task 1 of the Quantium Virtual
Internship. It will walk you through the analysis, providing the
scaffolding for your solution with gaps left for you to fill in
yourself.

Look for comments that say ``over to you'' for places where you need to
add your own code! Often, there will be hints about what to do or what
function to use in the text leading up to a code block - if you need a
bit of extra help on how to use a function, the internet has many
excellent resources on R coding, which you can find using your favourite
search engine. \#\# Load required libraries and datasets Note that you
will need to install these libraries if you have never used these
before.

\begin{Shaded}
\begin{Highlighting}[]
\DocumentationTok{\#\#\#\# Example code to install packages}
\CommentTok{\#install.packages("data.table")}
\DocumentationTok{\#\#\#\# Load required libraries}
\FunctionTok{library}\NormalTok{(}\StringTok{"data.table"}\NormalTok{)}
\FunctionTok{library}\NormalTok{(ggplot2)}
\FunctionTok{library}\NormalTok{(ggmosaic)}
\FunctionTok{library}\NormalTok{(readr)}
\DocumentationTok{\#\#\#\# Point the filePath to where you have downloaded the datasets to and }
\DocumentationTok{\#\#\#\# assign the data files to data.tables}
\CommentTok{\# over to you! fill in the path to your working directory. If you are on a Windows machine, you will need to use forward slashes (/) instead of backshashes (\textbackslash{})}
\NormalTok{filePath }\OtherTok{\textless{}{-}} \StringTok{""}
\NormalTok{transactionData }\OtherTok{\textless{}{-}} \FunctionTok{fread}\NormalTok{(}\FunctionTok{paste0}\NormalTok{(filePath,}\StringTok{"C:/Users/Luke Villanueva/source/repos/LukVill/Quantium Program/QVI\_transaction\_data.csv"}\NormalTok{))}
\end{Highlighting}
\end{Shaded}

\begin{verbatim}
## Taking input= as a system command ('C:/Users/Luke Villanueva/source/repos/LukVill/Quantium Program/QVI_transaction_data.csv') and a variable has been used in the expression passed to `input=`. Please use fread(cmd=...). There is a security concern if you are creating an app, and the app could have a malicious user, and the app is not running in a secure environment; e.g. the app is running as root. Please read item 5 in the NEWS file for v1.11.6 for more information and for the option to suppress this message.
\end{verbatim}

\begin{verbatim}
## Warning in (if (.Platform$OS.type == "unix")
## system else shell)(paste0("(", : '(C:/Users/Luke
## Villanueva/source/repos/LukVill/Quantium Program/QVI_transaction_data.csv)
## > C:\Users\LUKEVI~1\AppData\Local\Temp\Rtmpw5AwMV\file531c6841aac' execution
## failed with error code 1
\end{verbatim}

\begin{verbatim}
## Warning in fread(paste0(filePath, "C:/Users/Luke
## Villanueva/source/repos/LukVill/Quantium Program/QVI_transaction_data.csv")):
## File 'C:\Users\LUKEVI~1\AppData\Local\Temp\Rtmpw5AwMV\file531c6841aac' has size
## 0. Returning a NULL data.table.
\end{verbatim}

\begin{Shaded}
\begin{Highlighting}[]
\NormalTok{customerData }\OtherTok{\textless{}{-}} \FunctionTok{fread}\NormalTok{(}\FunctionTok{paste0}\NormalTok{(filePath,}\StringTok{"C:/Users/Luke Villanueva/source/repos/LukVill/Quantium Program/QVI\_purchase\_behaviour.csv"}\NormalTok{))}
\end{Highlighting}
\end{Shaded}

\hypertarget{exploratory-data-analysis}{%
\subsection{Exploratory data analysis}\label{exploratory-data-analysis}}

The first step in any analysis is to first understand the data. Let's
take a look at each of the datasets provided. \#\#\# Examining
transaction data We can use \texttt{str()} to look at the format of each
column and see a sample of the data. As we have read in the dataset as a
\texttt{data.table} object, we can also run \texttt{transactionData} in
the console to see a sample of the data or use
\texttt{head(transactionData)} to look at the first 10 rows. Let's check
if columns we would expect to be numeric are in numeric form and date
columns are in date format.

\begin{Shaded}
\begin{Highlighting}[]
\DocumentationTok{\#\#\#\# Examine transaction data}
\end{Highlighting}
\end{Shaded}


\end{document}
